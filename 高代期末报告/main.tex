%!TeX program = xelatex
\documentclass[12pt,hyperref,a4paper,UTF8]{ctexart}
\usepackage{CQUReport}

%%-------------------------------正文开始---------------------------%%
\begin{document}

%%-----------------------封面--------------------%%
\cover
\thispagestyle{empty} % 首页不显示页码



\newpage
%%------------------摘要-------------%%
\begin{abstract}

本文抛弃传统线性代数中线性空间八条公理。采用抽象代数的思想,从集合论开始,经过群论,自底向上构建体系,得到线性空间的定义。再证明线性空间满足那八条性质。

按照定义顺序,主要包括:集合,映射,笛卡尔积,运算,代数系统,群,阿贝尔群,环,域,线性空间。

\end{abstract}

\vspace{1cm}

%%--------------------------目录页------------------------%%
% \newpage
\tableofcontents

%%------------------------正文页从这里开始-------------------%
\newpage

%可选择这里也放一个标题
\begin{center}
   \title{ \Huge \textbf{{从集合论到线性空间}}}
\end{center}

\section{前言说明}
本文章使用 overleaf编写,借用了上交的\LaTeX 模板。除模板外的所有内容均为本人编写,也算是作者本人的对\LaTeX 书写论文的练习和熟悉。

作者本人仅仅为物理专业的普通大二学生,数学爱好者,非数学专业。文中难免出现错误和不严谨之处,忘多多包涵,有问题也欢迎批评指正。

本文并非是一篇论文,只是一次普通的期末报告。不会严格按照论文的格式书写。而是比较注重展示数学之美以及自认为的严谨性(所以会把参考文献写在文中,方便追根溯源)。

本文的写作主要参考《离散数学》,《量子计算:一种一种应用方法》,蓝以中的《高等代数》,丘维声的《高等代数》。

\section{集合论}

无论是高中还是大学,数学的第一门课几乎都是从集合开始的。有趣的是,由于罗素悖论的存在,对于集合的定义似乎并不完备,当然这并不是这篇文章想要讨论的。这里使用最传统的表述:

\tbox{一个集合是具有某种共同性质的东西组成的一个整体。}

\subsection{函数和映射}
这里的定义参考了丘维声的《高等代数》,并使用自己的语言进行简写。可能不够严谨。
\begin{Definition}
$f$是集合$S$到集合$S'$的一个映射
    $$f:S\rightarrow S'\Longleftrightarrow \forall a\in S , \exists \text{唯一} b\in S' ,f(a)=b
    $$
\end{Definition}

\tbox{在线征集“存在唯一”的符号表达()}

在传统意义上,当$S$和$S'$为数域时,我们称$f$为函数。

在《离散数学》的教材中,并不区分函数和映射(想来是因为计算机函数往往非常抽象,不仅和数域无关,甚至不是一个映射),秉承程序员的自我修养,本文把函数和映射统称映射。

关于映射,有一些基本的定义。
\begin{Definition}
映射$f$是单射,当且仅当
    $$\forall x_1,x_2\in S,x_1\ne x_2\Longleftrightarrow f\left( x_1 \right) \ne f\left( x_2 \right) $$
\end{Definition}

\begin{Definition}
映射$f$是满射,当且仅当
    $$\forall y\in S',\exists x \in S, y=f(x) $$
\end{Definition}

\begin{Definition}
映射$f$是双射,当且仅当$f$是满射又是单射
\end{Definition}

% \begin{Definition}
% 映射$f$是可逆的,当且仅当$f$是双射 %remake
% \end{Definition}

\subsection{笛卡尔积}

\begin{Definition}
笛卡儿积$A\times B$满足
$$
A\neq \oslash ,B\neq \oslash, A\times B:={(a,b)|a\in A,b\in B}.
$$
\end{Definition}

特别的

\begin{Definition}
$$    \begin{cases}
        A\times A :=A^2.\\
        A\times A^k :=A^{k+1}, k\in \mathbb{N}
    \end{cases}$$

\end{Definition}
这是一个递归定义,事实上,若$A$为数域,就可以定义传统的向量空间,但这并非本文想要讨论的。

\subsection{运算}
这里参考蓝以中的《高等代数》和《离散数学》,并进行改写。

\begin{Definition}
映射$f:A^n\rightarrow B$ 称为集合A上的一个n元运算
\end{Definition}

\begin{Definition}
映射$f:A\times A\rightarrow A$ 称为集合A内的一个代数运算
\end{Definition}

注意到,在后一个定义中,$f$是二元运算,且满足封闭性。

事实上蓝以中的教材中只给出了“代数运算”的定义,对于“运算”的定义来自《离散数学》。

接下来对代数运算作进一步讨论

\subsubsection{代数运算}\label{daishu}
设$*$是集合$R$内的一个代数运算,


\begin{Definition}
    若$\forall x,y \in R,x*y=y*x$称$*$满足交换律。
\end{Definition}

\begin{Definition}
    若$\forall x,y,z \in R,(x*y)*z=x*(y*z)$称$*$满足结合律。
\end{Definition}
    

\begin{Definition}
    若$\exists x_0 \in R,\forall x \in R,x_0*x=x_0$称$x_0$为零元,记为$0$
\end{Definition}


\begin{Definition}
    若$\exists x_1 \in R,\forall x \in R,x_1*x=x$称$x_1$为幺元(或单位元),记为$1$
\end{Definition}


\begin{Definition}
  若$\exists x_1,x_2 \in R,x_1*x_2=1$称$x_1,x_2$互为逆元,$x_1,x_2$的逆元分别记为$x_2^{-1},x_1^{-1}$  
\end{Definition}


\tbox{事实上,这里的集合$R$可以看作实数集$\mathbb{R}$,运算$*$可以看作实数的乘法$\times$,此时上述五条均满足}

其实更严谨的定义需要分左右讨论,即左零元、右零元,左幺元、右幺元,左逆元、右逆元。但可以证明其唯一性,故上述定义亦完备。



\section{群论}

以下论述主要参考于本人辅修计算机科学与技术专业时学的《离散数学》,教材作者是左孝凌、李为鉴、刘永才,1982年9月由上海科学技术文献出版社出版。我对定义使用当且仅当(iff)重新表述,从代数系统开始,逐渐添加限制条件,得到阿贝尔群,再借此得到环和域的定义,只认为颇具数学之美。有趣的事,在查阅资料的过程中,发现这些东西都来自于数学系的《抽象代数》,虽没学过,但看名字就知道不会简单,每念及之,无不佩服于数学的强大和泛用性。

\subsection{代数系统和群}

代数系统
\begin{Definition}

    一个非空集合A连同若干定义在A上的运算$f_1,f_2,\cdots ,f_k$所组成的系统,称为代数系统,记作$\left< A,f_1,f_2,\cdots ,f_k \right>  $
\end{Definition}

\tbox{值得一提的是,这里的定义中运算$f$并非代数运算,似乎有些文章中代数系统的定义会要求$f$满足封闭性,本文不要求,特此说明。}


广群
\begin{Definition}
$$
\text{代数系统}\left< S,* \right> ,\text{是广群,当且仅当}* \text{是代数运算}
$$
\end{Definition}

半群
\begin{Definition}
$$
\text{广群}\left< S,* \right> \text{是半群,当且仅当}*\text{满足结合律}
$$
\end{Definition}

独异点
\begin{Definition}
$$
\text{半群}\left< S,* \right> \text{是独异点(或含幺半群),当且仅当}S\text{关于}* \exists \text{幺元}
$$
\end{Definition}

群
\begin{Definition}
$$
\text{独异点}\left< G,* \right> \text{是群,当且仅当}\forall{x}\in G ,\exists x^{-1}
$$
\end{Definition}

阿贝尔群(交换群)
\begin{Definition}\label{base4}
$$
\text{群}\left< G,+ \right> \text{是阿贝尔群,当且仅当}+\text{满足交换律}
$$
\end{Definition}

在我写完这个部分后,发现似乎\ref{daishu}的内容应该放到代数空间里更为严谨,即在定义半群的同时定义幺元,定义独异点的同时定义逆元,定义阿贝尔群的同时定义交换律,等等。但我实在不愿破坏现在从广群到阿贝尔群的定义表述的连贯性,(我称其为语文之美)。
故不作更改,特此说明。

\subsection{环和域}

\textbf{环}
\begin{Definition}
$$
\text{代数系统}\left< A,+,* \right> \text{是一个环,当且仅当}\left< A,+ \right> \text{是阿贝尔群,}\left< A,* \right> \text{是半群}
$$
且$*$对于$+$是可分配的,即$\forall x,y,z\in A$,
$\begin{cases}
    x*(y+z)=x*y+x*z, \\
    (y+z)*x=y*x+z*x.
\end{cases}
$
\end{Definition}

\textbf{交换环}
\begin{Definition}
$$
\text{环}\left< A,+,* \right> \text{是交换环,当且仅当}\left< A,* \right> \text{是可交换的}
$$
\end{Definition}

\textbf{含幺环}
\begin{Definition}
$$
\text{环}\left< A,+,* \right> \text{是含幺环,当且仅当}\left< A,* \right> \text{是独异点}
$$
\end{Definition}

\textbf{整环}
\begin{Definition}
$$
\text{环}\left< A,+,* \right> \text{是整环,当且仅当}\left< A,+,* \right> \text{是含幺交换环}
$$
且$\left< A,* \right>$无零因子,即$\forall x,y\in A$,$x\neq 0 , y\neq 0 \Rightarrow x*y\neq 0$

\end{Definition}

\textbf{域}
\begin{Definition}
$$
\text{整环}\left< A,+,* \right> \text{是域,当且仅当}\left< A-\{0\},* \right>\text{是阿贝尔群}
$$
\end{Definition}

对于环的域这一节的表述没有群那一节那么美观,我已尽力简化,但依然无法一气呵成,可能是我能力不够吧。

\section{线性空间}

关于线性空间和向量空间的关系需要先说明一下。在蓝以中的教材中,先是定义了向量空间,再定义线性空间,并认为线性空间包含向量空间,这当然没有问题,毕竟线性空间的元素不一定是向量。而在丘维声的教材中,有这么一句话:\textbf{借助几何语言,把线性空间的元素称为向量,线性空间又可称为向量空间}。这种说法同样没有问题,毕竟二者也确实等价。在本文中,我并没有引入向量的概念,故采用后者的说法,即认为向量空间和线性空间是一个东西的两个名称。


\subsection{八条公理}

\begin{Definition}\label{base1}
域$\mathbb{F}$上的一个线性空间V由一个阿贝尔群V以及一个域$\mathbb{F}$在V上的数乘运算构成
\end{Definition}

\tbox{其中只用集合的符号,即单个大写字母表示群和域,特此声明。}

我看到这个定义\ref{base1}是在《量子计算:一种应用方法》这本书中,作者是杰克-希德里,人民邮电出版社出版。我正是因为看了这本书,才萌生了写这篇文章的想法。用一句话便定义了线性空间,是如此的简洁优美。

再看看传统的定义是多么的丑陋。

\begin{Definition}
$$
\begin{aligned}
&\text{ 设V 是一个非空集合,K 是一个数域.又设:} \\
&\mathrm{(i)}\text{ 在}V\text{ 中定义了一种运算,称为加法. 即对}V\text{ 中任意两个元素} \\
&\alpha 与 \beta,\text{都按某一法则对应于}V\text{ 内唯一确定的一个元素,记之为} \alpha+\beta; \\
&\mathrm{(ii)}\text{ 在 K 中的数与}V\text{ 的元素间定义了一种运算},\textbf{称为数乘}.\\ 
&\text{即} \text{对} V\text{中任意元素 }\alpha\text{ 和数域 }K\text{ 中任意数 }k,\\
&\text{都按某一法则对应于 }V\text{ 内}  
\text{唯一} \text{确定的一个元素,记之为 }k\alpha.  
\end{aligned}$$
如果加法与数乘满足下面列出的八条运算法则,那么称V 是数域K 上的一个线性空间. 

加法和数乘满足下面的八条运算法则:
$$\begin{aligned}\label{base2}
& \text{(i)对任意 }\alpha,\beta,\gamma\in V,\alpha+(\beta+\gamma)=(\alpha+\beta)+\gamma;  \\
&\text{(ii)对任意 }\alpha,\beta{\in}V\text{,}\alpha+\beta{=}\beta{+}\alpha; \\
& (\text{iii)存在一个元素 }0\in V,\text{使对一切 }\alpha{\in}V,\text{有} \alpha+0=\alpha, \\
&\text{此元素0称为V 的零元素;}  \\
&(\text{iv)对任一 }\alpha{\in}V\text{ 都存在 }\beta{\in}V,\text{使} \alpha+\beta=0, \\
&\beta  \text{称为}\alpha \text{的一个负元素;}  \\
&\mathrm{(v)}\text{ 对数域中的数 }1,\text{有 }1\bullet\alpha{=}\alpha; \\
&(\mathrm{vi})\text{ 对任意 }k,l\in K,\alpha{\in}V,\text{有} (kl)\alpha  =k\left(l\alpha\right);  \\
&(\text{vii)对任意 }k,l\in K,\alpha\in V,\bar{\lambda} \text{有}  (k+l)\alpha  =k\alpha+l\alpha;  \\
&(\text{viii)对任意 }k\in K,\alpha,\beta{\in}V \text{,有}  k\left(\alpha+\beta\right) =k\alpha+k\beta. 
\end{aligned}$$
\end{Definition}

定义\ref{base2}来自于蓝以中的《高等代数》。为方便讨论,简写为下面的定理。

\begin{Theorem} \label{level 1}
    线性空间满足下列八条性质
\begin{gather}
    \boldsymbol{a}+\boldsymbol{b}=\boldsymbol{b}+\boldsymbol{a}
\\
\boldsymbol{a}+\boldsymbol{b}+\boldsymbol{c}=\left( \boldsymbol{a}+\boldsymbol{b} \right) +\boldsymbol{c}=\boldsymbol{a}+\left( \boldsymbol{b}+\boldsymbol{c} \right) 
\\
\boldsymbol{a}+\mathbf{0}=\boldsymbol{a}
\\
\boldsymbol{a}-\boldsymbol{a}=\boldsymbol{a}+\left( -\boldsymbol{a} \right) =\mathbf{0}
\\
1\cdot \boldsymbol{a}=\boldsymbol{a}
\\
\lambda \left( \mu \boldsymbol{a} \right) =\left( \lambda \mu \right) \boldsymbol{a}
\\
\lambda \left( \boldsymbol{a}+\boldsymbol{b} \right) =\lambda \boldsymbol{a}+\lambda \boldsymbol{b}
\\
\left( \lambda +\mu \right) \boldsymbol{a}=\lambda \boldsymbol{a}+\mu \boldsymbol{a}
\end{gather}

这个定理的表述是我直接从网上copy下来的。并不严格,比如压根没说
$a,b,c,+,$
$*,\lambda,\mu$ 是什么,但这些都是我们很熟悉的东西,非常显然,在这里省略亦无伤大雅。
\end{Theorem}

注意到定理\ref{level 1}我用了\textbf{定理}而不是\textbf{公理}。那既然是\textbf{定理}当然需要证明了。而使用定义\ref{base1}证明定理\ref{level 1}正是这篇文章的最终目的。

但在开始之前还有一个东西没有定义,即域$\mathbb{F}$在V上的\textbf{数乘运算}。
参考《量子计算》

\begin{Definition}\label{base3}
    域$\mathbb{F}$在V上的\textbf{数乘运算}$*$是映射$F\times V\rightarrow V$满足

    \begin{itemize}
        \item 分配律1:$\forall a\in F,\forall u,v \in V$有
        $$a*(u+v)=a*u+a*v$$
        \item 分配律2:$\forall a,b\in F,\forall v \in V$有
        $$(a+b)*v=a*u+b*v$$
        \item 结合律:$\forall a,b\in F,\forall v \in V$有
        $$(ab)*v=a*(b*v)$$
        \item 存在单位元:$\forall v \in V,\exists 1 \in F$有 
        $$1*v=v$$
    \end{itemize}
\end{Definition}

值得注意的是,虽然我只使用了加法$+$和数乘$*$,但上述定义一共包含了四个运算,
展开之后很容易理解

域$\mathbb{F}$在V上的\textbf{数乘运算} $\Longleftrightarrow$
域$\left< F,+_1,\times \right>$在阿贝尔群$\left< V,+_2 \right>$上的\textbf{数乘运算}$*$.

他们分别是F上的加法,F上的乘法,V上的加法,F和V上的乘法。它们对应于不同的映射,完全不同,但性质又是如此相似,并能彼此联系在一起,构成一个具有强大泛用性的系统。在我看来这正是数学美的所在。

现在完事具备,只欠证明,事实上,定理\ref{level 1}已经非常显然了。

\begin{proof}
    $\because $由$\left<V,+\right>$是阿贝尔群,由定义\ref{base4}直接得到前四条性质,其中V关于$+$的逆元定义为负元。
    
    由数乘运算的定义\ref{base3},直接得到后四条性质。
\end{proof}

最后,有一个非常有趣的问题
\begin{Proposition}
    定义\ref{base1}与定义\ref{base2}不等价。
\end{Proposition}

这个命题的表述是我自己想的,可能不正确。不过丘维声的定义就不存在这个问题
\begin{proof}
    因为定义\ref{base1}并没有要求域$\mathbb{F}$是数域,而定义\ref{base2}要求。所以原命题得证。
\end{proof}

至此,所有正文内容就结束了,后面是一些简单的性质和证明。也是对\ref{daishu}的补充。


\subsection{性质}

\begin{Proposition}
    零元是唯一的
\end{Proposition}

\begin{proof}

\[
\text{设} \exists \mathbf{0}_1,\mathbf{0}_2 ,\forall x\in V,0_1+x=x,0_2+X=X
\]
那么
$$0_2=0_1+0_2=0_2+0_1=0_1$$
所以零元唯一
\end{proof}

\begin{Proposition}
负元是唯一的
\end{Proposition}

\begin{Proposition}
$$\forall \alpha\in V , 0*\alpha=0$$
\end{Proposition}

\begin{Proposition}
k*0=0
\end{Proposition}

\begin{proof}

\begin{align*}
&k*0=k*(0+0)=k*0+k*0\\
&k*0+(-k*0)=k*0+k*0+(-k*0)\\
&0=k*0+0\\
&0=k*0
\end{align*}

\end{proof}

\begin{Proposition}
$$k*\alpha=0\Longleftrightarrow k=0 \,\text{or}\, \alpha=0$$
\end{Proposition}

\begin{proof}
$\Leftarrow$ 已证明

$\Rightarrow$

假设 $k!=0$ and $\alpha!=0$

$$\exists k^{-1},k^{-1}*k=1$$

then

$$\alpha =1*\alpha=(k^{-1}k)\alpha=k^{-1}(k*\alpha)=0$$

矛盾
\end{proof}

\begin{Proposition}
$$(-1)*\alpha=-\alpha$$
\end{Proposition}

\begin{proof}
$$  \begin{cases}
       \alpha+(-1)*\alpha=(1+(-1))*\alpha=0*\alpha=0
    \end{cases}
$$

根据负元的定义,原命题得证
\end{proof}

基变换

\begin{gather*}
\text{对}\mathbf{R}^n\text{的两个基}\\
\mathrm{I} : \left\{ \bm{b}_1,\bm{b}_2,\cdots ,\bm{b}_n \right\} \\
\mathrm{II} : \left\{ \bm{f}_1,\bm{f}_2,\cdots ,\bm{f}_n \right\} \\
\text{构造两个基矩阵}\\
\bm{B}={\mathrm{def}}\left[ 
\begin{matrix}
	\bm{b}_1&		\bm{b}_2&		\cdots&		\bm{b}_n\\
\end{matrix} \right] \\
\bm{F}={\mathrm{def}}\left[ \begin{matrix}
	\bm{f}_1&		\bm{f}_2&		\cdots&		\bm{f}_n\\
\end{matrix} \right] \\
\text{称}\bm{F}^{-1}\bm{B}\text{为} \mathrm{I} \text{到} \mathrm{II} \text{的过渡矩阵,而}\bm{B}^{-1}\bm{F}\text{为} \mathrm{II} \text{到} \mathrm{I} \text{的过渡矩阵,过渡矩阵关联式}\\
\bm{v}_{\bm{F}}=\bm{F}^{-1}\bm{Bv}_{\bm{B}}\\
\bm{v}_{\bm{B}}=\bm{B}^{-1}\bm{Fv}_{\bm{F}}\\
\text{称为坐标变换公式}\\
\mathrm{II}:\left\{ \bm{f}_1,\bm{f}_2,\cdots ,\bm{f}_n \right\} \\
\text{构造两个基矩阵}\\
\bm{B}={\mathrm{def}}\left[ \begin{matrix}
	\bm{b}_1&		\bm{b}_2&		\cdots&		\bm{b}_n\\
\end{matrix} \right] \\
\bm{F}={\mathrm{def}}\left[ \begin{matrix}
	\bm{f}_1&		\bm{f}_2&		\cdots&		\bm{f}_n\\
\end{matrix} \right] \\
\text{称}\bm{F}^{-1}\bm{B}\text{为I到II的过渡矩阵,而}\bm{B}^{-1}\bm{F}\text{为II到I的过渡矩阵,过渡矩阵关联式}\\
\bm{v}_{\bm{F}}=\bm{F}^{-1}\bm{Bv}_{\bm{B}}\\
\bm{v}_{\bm{B}}=\bm{B}^{-1}\bm{Fv}_{\bm{F}}\\
\text{称为坐标变换公式}
\end{gather*}

基变换放在这里非常突兀,其实我在这里只是想借此表达一个观点,那就是\textbf{矩阵是通过基变换定义的},再之后\textbf{行列式可以通过矩阵定义}。由此就能构建高等代数的体系。

\tbox{而不是像几乎所有的教材上,一上来就定义行列式}

\section{后记}

\subsection{关于本文}
事实上,本文中的大部分内容并非高等代数课上学到的,它们主要来自于本人在学习《量子计算:一种应用方法》和《离散数学》时的一些思考,正巧高代期末要写期末报告,便决定以此为主题。当然,在撰写文章的时候我也参考了蓝以中和丘维声的《高等代数》教材。我发现蓝以中和丘维声都对集合论,笛卡尔积,包括群(群只有蓝以中的书中有)、环、域的概念有一些论述。但基本上都是运算法则的罗列,并没有像本文这样自底向上构建系统。当然,它们毕竟是《高等代数》而不是《抽象代数》,没必要这么“抽象”。我自然没想和他们比,而且我想几百年前肯定就有人做过我现在正在的事了。更何况丘维声也写了一本专门的《近世代数》,不过我并没有看过。不过,这个过程亦是有趣的。使用\LaTeX 书写文章和公式也颇具美感。乐亦在其中也。而且这应该是我人生中的第一篇像论文一样的文章吧,满满的成就感。

\subsection{关于高代学习}
我《高等代数》的学习充满艰辛与乐趣。沉迷于抽象的数学表达和逻辑证明,不是很喜欢机械的计算,比如高斯消元、行列式和特征值等。因为我觉得机械的计算都是可以通过计算机实现的,像matlab、mathmatica之类强大的软件,人算得好没有太大意义,再好也比不过计算机。但数学定理的证明,公理体系的建立却是非人类莫属的。(严谨一点,不得不提,计算机也能辅助证明,而且历史悠久,比如大名鼎鼎的四色问题,另外男神陶哲轩也写过一篇文章,做过一个报告,专门讲这个事)。最后将一个故事,我上个星期参加计算机学院组织的“挑战杯”程序设计算法竞赛时,遇到了一个题,就是计算一个普普通通的范德蒙行列式,很容易求出通解。

\[
\begin{vmatrix}
1 & 1 & 1 & \cdots & 1 \\
a_1 & a_2 & a_3 & \cdots & a_n \\
a_1^2 & a_2^2 & a_3^2 & \cdots & a_n^2 \\
\vdots & \vdots & \vdots & \ddots & \vdots \\
a_1^{n-1} & a_2^{n-1} & a_3^{n-1} & \cdots & a_n^{n-1}
\end{vmatrix}
= \prod_{1 \leq i < j \leq n} (a_j - a_i)
\]

如果使用计算机直接套公式,在n太大的情况下,复杂度太高,会超时。但是如果经过一系列变形,写成一个递推的形式。再利用线段树这种高级数据结构,便能在规定的时间里计算出结果。具体算法这里不提。只想表达一个观点:

\textbf{只会计算学不好数学,但不会计算更学不好数学。}

\subsection{发布地址}
你能在下面的地址找到\LaTeX 源文件。
\begin{itemize}
    \item GitHub: \url{https://github.com/XeriChen/advanced_algebra_final_assignment}
    % \item Overleaf:  \url{https://www.overleaf.com/latex/}
\end{itemize}

\vspace{1cm}
\Huge{\textbf{感谢您能看到最后。}}
%%----------- 参考文献 -------------------%%
%在reference.bib文件中填写参考文献,此处自动生成

%\reference



\end{document}